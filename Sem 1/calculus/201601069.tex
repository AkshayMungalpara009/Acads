% MATHS DIWALI ASSIGNMENT 

% Amey Ghate 
% 201601069

\documentclass[12pt]{article}
\usepackage{graphicx}
\usepackage{times}

\usepackage{type1cm}
\usepackage{eso-pic}

\usepackage{color}
\usepackage{ulem}
\usepackage{everypage}
\usepackage{amssymb}
\usepackage{draftwatermark}
% Use the following to make modification
\usepackage[margin=3 cm]{geometry}
%packages added
\usepackage{amsmath}

%\usepackage[usenames,dvipsnames]{color}
\begin{document}

\SetWatermarkAngle{45}
\SetWatermarkLightness{0.95} 
\SetWatermarkFontSize{1cm}
\SetWatermarkScale{5}
\SetWatermarkText{Amey Ghate}


% Article top matter
\title{\textcolor{blue}{Calculus used in Defence Strategy}}

\begin{center}
\Huge
{\bf \textcolor{blue}{ 
{ DIWALI ASSIGNMENT ON CALCULUS} \\
}
}
\vspace{2 cm}
\huge\bf
{ Assigned by : Professor Manish K Gupta \\
COURSE : SC 107 \\
Calculus with complex variables \\
FALL 2016 \\
DA-IICT \\
GANDHINAGAR \\
}
\vspace{5 cm}
\LARGE {\it
{ 
Made by: Amey Ghate \\
ID : 201601069 \\
}
}
\end{center}
\author{
Amey Ghate\\
201601069,\\
DAIICT,\\
GANDHINAGAR,\\
INDIA\\
\texttt{201601069@daiict.ac.in}
} 
\date{\today}
%\today is replaced with the current date

\maketitle

\begin{abstract}
\textcolor{blue}
{
\large
{
In this document, I shall discuss how to use calculus in defining country's
\uwave {`competitive defence strategy'}.
We will see the mathematical model for rate of growth of economy
and impact of defence expenditure. \uwave {`How country's economy can get devastated'} 
due to over defence spending. 
}
}
\end{abstract}

\begin{center}
\includegraphics{201601069_daiict.png}
\end{center}

\newpage
\large
\section{\textcolor{blue}{INTRODUCTION}}


The amount of money which a country can invest on defence is determined by the size of its economy
i.e. (GDP of that country). Bigger the economy, greater is the country’s ability to spend on defence.
A country can use economic warfare as a strategy to weaken the economy of another country.

Two hostile countries have competition between them to have the best armed forces each country
competes to produce larger number of weapons, greater army and superior military technology.
This causes strain on economy of both the countries. In this document we will see how a faster growing 
economy can devastate any country by just increasing defence expenditure.

Details of economic warfare and arms race is available at: \cite{1} \cite{2}

\section{\textcolor{blue}{ MATHEMATICAL MODEL }}
%\textsf
A mathematical model for rate of growth of economy w.r.t. economy of the country and its defence 
expenditure is discussed in \cite{3}. It can be simplistically represented as 

$Rate of growth = constant*current GDP – defence expenditure$
\vspace{.5 cm}

\textcolor{blue}
{
$f'(t) = k f(t) - d(t)$ \\
where: 
\begin{align*}
f(t) &= \textit{current GDP} \\
t &= \textit{time measured in years} \\
f’(t) &= \textit{rate of change of GDP w.r.t. time} \\
k &= \textit{constant – coefficient determining the pace of economic growth} \\
d(t) &= \textit{defence expenditure} 
\end{align*}
}
%\vspace{.25 cm}

defence expenditure of a country is considered to be a fraction of its GDP. Thus the equation will become 
$d(t) = q* f(t)$ here 'q' is the percentage of GDP being spent on defence. 

Now substituting the values in $f’(t) = k*f(t) - d(t)$ and integrating we get
$f(t) = f(0)*e^{(k-q)t}$ here in the above equation f(0) is the value of GDP at t=0.

\newpage
\section{\textcolor{blue}{REAL LIFE SITUATION}}
\textsf
Country A and Country B are competing with each other and are in economic warfare. Both the 
countries want to keep balance of power hence they are bound to invest equally in defence.
Country A and B have 10 trillion GDP initially.
Country A has 10 percent growth rate whereas country B has 2.5 percent growth rate
Country A decides to invest 4 percent of GDP into Defence. Despite weak growth rate
Country B is forced to invest 4 percent into Defence.
Now Country A is in dominating position to decide its Defence expenditure strategy and
economically ruin Country B.


\section{\textcolor{blue}{USE OF MATHEMATICAL MODEL TO REAL LIFE SITUATION}}
\textsf
Let us know understand how country A can use Calculus to economically devastate country B

Consider country A having its GDP $A(t) = A(0)*e^{(k-a)t}$
Let 10 trillion be initial GDP and 10 percent is the growth rate, let 4 percent be the expenditure on the defence.

$A(t) = 10 trillion * e^{(10-4)t}$

Country B has to match defence expenditure and assume it’s growth rate is 2.5 percent

$B'(t) = l*B(0) - a * A * e^{(k-a)t}$ ,

here I is the growth rate of country and B(0) = 10 trillion
Integrating both sides of $B'(t) = l*B(0)-{a*A*e}^{(k-a)t}$ from 0 to t, we get 

\textcolor{blue}
{
\[
\mathbf
{
B(t) = e^{lt}[B(0)-\int{a*A(0)e^{(k-a-l)*t}}] \\
}
\]
}
Wherein \\
\textcolor{blue}
{
\[
\mathbf
{
B(t) = e^{l*t}{B(0)-(a*A(0)[e^{(k-a-l)t}-1]/(k-a-l))} \\
}
\]
}
Putting the values now – 
\[
\textcolor{blue}
{
\mathbf
{
B(t) = e^{2.5t}{10 trillion – (4*10 trillion[e^{3.5t} - 1]/3.5)}
}
}
\]

Country B will get economically devastated when its GDP becomes "0" 
due to the defence expenditure and the equation becomes - 
\[
\textcolor{blue}
{
\mathbf
{
0 = B(0) – a*A(0)*[e^{(k-a-l)t}-1]/(k-a-l))
}
}
\]
Solving above equation we get 
\[
\textcolor{blue}
{
\mathbf
{
T = ln[1 + [(k-a-l) *B(0) /( a*A(0))]*1/(k-a-l)
}
}
\]
Solving this we understand that country B does not have economic existence after 17 years.

\section{\textcolor{blue}{CONCLUSION}}
\textsf
{
Calculus can be used in determining defence expenditure and strategy. Dominant nation can actually plan economic fall of competing nation. This is how
Country A adopted defence investment to economically devastate County B
}
\bibliographystyle{plain}
\bibliography{201601069}

\end{document} 
%End of document.
